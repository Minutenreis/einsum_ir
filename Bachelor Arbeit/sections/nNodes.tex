\section{Scaling up the number of nodes}

The previously discussed master-worker algorithm expects all input tensors before and the output tensor after a contraction to reside within a single master node.
Imagine we now have $p$ nodes (this notation for the number of nodes will hold true throughout this chapter).
The master node is now involved in sending and receiving $\frac{p-1}{p}$ parts of the tensors in the contraction.
Meanwhile the worker nodes each only communicate $\frac{1}{p}$ part of the tensors with the master.
The larger $p$ gets the more unbalanced communication in the master-worker algorithm gets.
It also might run into memory issues should the total size of all tensors exceed the master nodes memory size.

Instead let us consider algorithms that work on distributed tensors as input and output.
Such algorithms allow the communication load to be balanced across all nodes.

As basis for such an algorithm, we make the following assumptions:
\begin{enumerate}
    \item input tensors are predistributed across all nodes
    \item output tensors may stay distributed
    \item tensors are sufficiently compute intensive that communication is not a bottleneck
    \item the distributed dimension is a multiple of the total number of nodes
    \item each node has multiple threads.
\end{enumerate}

% todo: justify assumptions after talking about algorithms

With those assumptions we will conceive three algorithms:

\subsection{Distributed c Dimension}

If a tensor contraction $A,B \rightarrow C$ shares a dimension between all three tensors, we will call it a c dimension.
Imagine $A$,$B$ and $C$ share such a c dimension $c_0$
Contracting such an expression is the same as splitting $A$ and $B$ alongside $c_0$ into $p$ equal chunks $A_0\dots A_{p-1}, B_0\dots B_{p-1}$, contracting the chunks $A_i, B_i \rightarrow C_i$ and then concatenating the $p$ chunks of $C_i$ at $c_0$ dimension to $C$.
Since the contractions $\forall i \in p: A_i, B_i \rightarrow C_i$ are independent of each other, we can work on them in parallel.

\subsection{Distributed m/n Dimensions}

\begin{algorithm}[h]
        \begin{algorithmic}
        \Require $i = \texttt{mpi\_rank}, A_i, B_i$
        \Ensure $C_i$
        \State $\texttt{next} = (i+1) \mod p$
        \State $\texttt{prev} = |(i-1) \mod p|$
        \State $\texttt{comp\_buffer} \gets B_i$
        \State  $\text{for } j \text{ in } 0\dots p \text{ do:}$
        \State \indent $k \gets (i + j) \mod p$
        \State \indent \text{do in parallel:}
        \State \indent \indent $A_i, \texttt{comp\_buffer} \rightarrow C_{i,j}$
        \State \indent \indent $\texttt{mpi\_recv recv\_buffer from next}$
        \State \indent \indent $\texttt{mpi\_send comp\_buffer to prev}$
        \State $C_i \gets \texttt{concat}(C_{i,0}\dots C_{i,p})$
    \end{algorithmic}
    \caption{Distributed m/n contract pseudocode}
    \label{m_n_pseudocode}
\end{algorithm}

If a tensor contraction $A,B \rightarrow C$ shares a dimension between one input tensor and the output tensor, but not the second input tensor, we will call it a m dimension if its included in the left input tensor and a n dimension if its included in the right one.
Imagine $A$ and $C$ share a m dimension $m_0$ and $B$ and $C$ share a n dimension $n_0$.
Contracting an expression $A,B \rightarrow C$ is then the same as splitting $A$ along $m_0$ into $p$ chunks $A_0\dots A_{p-1}$ and $B$ along $n_0$ into $p$ chunks $B_0\dots B_{p-1}$.
We can imagine $C$ then as the concatination of $C_0\dots C_p$ chunks split along the $m_0$ dimension and $C_i$ for $0 \leq i \leq p$ as the concatination of $C_{i,0} \dots C_{i,p}$ along $n_0$.
We can compute each $C_{i,j}$ as $A_i,B_j \rightarrow C_{i,j}$.
To calculate a chunk $C_i$ a process needs all of $A$.
Instead of executing an \texttt{allgather} on $A$ we will calculate $C_i$ stepwise.
In each step each node will contract one part $C_{i,j}$ of $C_I$ as $A_i,B_j \rightarrow C_{i,j}$.
Simultaneously each node will get a new chunk $B_{j+1}$ from their neighbour with a higher rank in a ring, so the node with the highest rank gets its chunk from the node with rank 0.
To enable both the contraction and the communication of the new chunk to occur simultaneously each node needs extra memory with the size of one chunk of $B$.
Since we assume all tensors to be predistributed, each node will begin with its diagonal part $C_{i,i}$ where i equals the rank of the node.
Following the above algorithm each node will compute $C_{i,i}, \dots C_{i,p-1},C_{i,0},\dots,C_{i,i-1}$.
After we concatenate those chunks $C_{0,i}\dots C_{p-1,i}$ to $C_i$ our algorithm is done.

In the implementation we can make the last concatenation implicit as long as we already calculate each chunk $C_{i,j}$ already with the correct strides in the results tensor $C_i$.
Due to limitations in the binary contraction interface of \texttt{einsum\_ir} however this is only possibly should each chunk $C_{i,j}$ be continuous in $C_i$.
So for the implementations we have to assume that $n_0$ is the outermost dimension of $C_i$.

We could also consider another very similar algorithm where each node gets a chunk $A_i$ in each iteration instead of $B_j$.
Such an algorithm is omitted here, since the same effect can be achieved by swapping the input tensors as $A' \coloneqq B$ and $B' \coloneqq A$, which results in $A',B' \rightarrow C$.