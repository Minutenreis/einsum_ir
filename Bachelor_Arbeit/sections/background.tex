\section{Background}

\subsection{Tensor Contractions}
\label{sec:tensorContractions}

%TODO: HOW THE HELL DO I INTRODUCE MATH?!
Tensor contractions are a generalization of batched matrix multiplications on n-dimensional arrays called order-n tensors.
In a binary contraction of two tensors $A$ and $B$ to a third tensor $C$ we refer to the first input tensor $A$ as left tensor and the second input tensor $B$ as the right one. %todo: will hier in Ordnung?
We use the following classification for the dimensions in the tensors $A$,$B$ and $C$:

\begin{tabular}{ |c|c|c|c| } 
  \hline
  $dimension type$ & present in A & present in B & present in C\\
  \hline
  $c$ & $\checkmark$& $\checkmark$& $\checkmark$\\
  $m$ & $\checkmark$& $\crossmark$& $\checkmark$\\
  $n$ & $\crossmark$& $\checkmark$& $\checkmark$\\
  $k$ & $\checkmark$& $\checkmark$& $\crossmark$\\
  \hline
\end{tabular}

We will further define two operations for our distribution.
The first operation is \textit{cutting} a tensor along a dimension $o$.
It shall refer to dividing a tensor into $n \in \mathbb{N}$ chunks.
The first such chunk composes the all values of the tensor with $o \in [1,\frac{|o|}{T}]$, the second all values with $o \in [\frac{|o|}{T}+1, 2 \cdot \frac{|o|}{T}]$ and so on.
The last chunk shall consist of all remaining values with $o \in [(n-1) \cdot \frac{|o|}{T}, |o|]$.
The second operation is \textit{concatenating} a tensor as the inverse operation of \textit{cutting}.


\subsection{Einsum Expressions}
\label{sec:einsum_expr}

Einsum expressions allow for a more succinct expression than the typical Tensor notation as used in \ref{sec:tensorContractions}.
Instead of writing $C_{pstuv}=\sum_{q}\sum_{r}A_{pqrs}B_{tuqvr}$ the same contraction is expressed as $A_{pqrs}B_{tuqvr} \rightarrow C_{pstuv}$.
The summation signs are now implicit.
Einsum expressions can also describe the contractions of more than 2 tensors\cite{einsum_is_all_you_need}.
%todo: can I just recommend people to look at resources such as "Einsum is all you need" for an advanced introduction?
Instead of writing $D_{ij} = \sum_{k}\sum_{l}A_{ik}B_{jkl}C_{il}$ it is expressed as $A_{ik}B_{jkl}C_{il} \rightarrow D_{ij}$.
All expression in the rest of this thesis are written as simplified einsum expression, either written as $A,B\rightarrow C$, leaving out the indices, or as $pqrs,tuqvr \rightarrow pstuv$, leaving out the tensor variables.

We will make a few observations about the different dimension types introduced in \ref{sec:tensorContractions} in the einsum notation for the general expression $A,B \rightarrow C$:

Assuming that $A$,$B$ and $C$ share a c dimension $c_0$ the expression is equivalent to:
\begin{enumerate}
  \item cut $A$,$B$ and $C$ along $c_0$ into $T \in \mathbb{N}$ chunks $A_1 \dots A_T$,$B_1 \dots B_T$ and $C_1 \dots C_T$
  \item $\forall i \in [1,T]: A_i, B_i \rightarrow C_i$
  \item concatenate all chunks $C_i$ along $c_0$ to $C$
\end{enumerate}

Assuming that $A$ and $C$ share an m dimension $m_0$ and $B$ and $C$ share an n dimension $n_0$ the expression is equivalent to:
\begin{enumerate}
  \item cut $A$ and $C$ along $m_0$ into $T \in \mathbb{N}$ chunks $A_1 \dots A_T$ and $C_1 \dots C_T$
  \item cut $B$ and all chunks $C_i$ along $n_0$ into $T$ chunks $B_1 \dots B_T$ and $C_{i,1} \dots C_{i,T}$
  \item $\forall i,j \in [1,T]: A_i, B_j \rightarrow C_{i,j}$
  \item $\forall i \in [1,T]:$ concatenate all chunks $C_{i,j}$ along $n_0$ to $C_i$
  \item concatenate all chunks $C_i$ along $m_0$ to $C$
\end{enumerate}

Assuming that $A$ and $B$ share a k dimension $k_0$ the expression is equivalent to:
\begin{enumerate}
  \item cut $A$ and $B$ along $k_0$ int $T \in \mathbb{N}$ chunks $A_1 \dots A_T$ and $B_1 \dots B_T$
  \item create $T$ tensors of the same shape as $C$ that we call $C_1 \dots C_T$
  \item $\forall i \in [1,T]: A_i, B_i \rightarrow C_i$
  \item add all tensors $C_i$ elementwise to $C$
\end{enumerate}


\subsection{einsum\_ir}
\label{sec:einsum_ir}

\texttt{einsum\_ir}\cite{einsum_ir} is a software to evaluate a series of einsum expressions expressed in tree form.
An example for an einsum tree is $[A,B\rightarrow C],D \rightarrow E$.
My work builds on top of this software, using their implementation of a binary tensor contraction $A,B \rightarrow C$ as local primitive for my algorithm.
\texttt{einsum\_ir} already implements shared memory parallelization with OpenMP\cite{openMP}, so it can already exploit all CPU cores on a single NUMA domain.
An important note for the current binary tensor contraction interface is that it expects all tensors, both input and output, to be stored in contiguous memory.

\subsection{MPI}

As a tool for distributed memory parallelization we will use MPI.
It is a standard for distributed memory parallelization libraries like OpenMPI and MPICH.
We will only use point-to-point communication where one process sends data directly to another.